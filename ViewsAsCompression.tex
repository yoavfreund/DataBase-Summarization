\documentclass[11pt]{article}
\usepackage{amsfonts}
\usepackage{amssymb,amsmath}
\usepackage{fullpage}
\usepackage{epsfig}
\usepackage[usenames]{color}
\setlength{\oddsidemargin}{0pt}
\setlength{\evensidemargin}{0pt}
\setlength{\textwidth}{6.0in}
\setlength{\topmargin}{0in}
\setlength{\textheight}{8.5in}

\newtheorem{Thm}{Theorem}
\newtheorem{Lem}[Thm]{Lemma}
\newtheorem{Cor}[Thm]{Corollary}
\newtheorem{Prop}[Thm]{Proposition}
\newtheorem{Claim}[Thm]{Claim}
\newtheorem{lemma}{Lemma}
\newtheorem{Observation}{Observation}
\newtheorem{theorem}{Theorem}
\newenvironment{proof}{\noindent {\sc Proof:}}{$\Box$ \medskip}

\newtheorem{Ex}{Exercise}
\newtheorem{Exa}{Example}
\newtheorem{Rem}{Remark}
\newtheorem{assumption}{Assumption}

\newcommand{\calA}{{\mathcal{A}}}
\newcommand{\calP}{{\mathcal{P}}}
\newcommand{\kc}[1]{\noindent {\color{red}{{\smallskip \textbf{KC: }}#1}}}
\newcommand{\djh}[1]{{\color{cyan} \{\textbf{YF: }#1\}}}
\newcommand{\bbR}{{\mathbf{R}}}
\newcommand{\disth}[1]{d_{\operatorname{ham}}(#1)}
\newcommand{\distp}[1]{d_{\operatorname{prok}}(#1)}
\newcommand{\calX}{{\mathcal{X}}}
\newcommand{\calF}{{\mathcal{F}}}
\newcommand{\bbE}{{\mathbb{E}}}

\newcommand{\err}{\mathbf{err}}
\newcommand{\calH}{\mathcal{H}}

\title{Approximate Views and compression}

\begin{document}

\maketitle

\section{Motivation}
We consider situations where we have a sensor network that is
distributed over a large area. Our goal is to perform continuous
analysis of the data generated by the sensors. The goal of the
analysis is to detect abnormal behaviour and to create predictive
models of the behaviour of the physical system (PS) that is monitored
by the sensors.

We assume the following generic architecture, consisting of sensor
nodes and compute nodes. Sensor nodes consist of some sensors a
general purpose computer and local storatge. Compute nodes consist
only of computers, potentially stronger than the computers in the
sensor nodes. The compute nodes communicate with each other and with
the sensor nodes. The compute nodes aggregate information from many
sensors in order to estimate the global state of the PS,
and to create predictive models for the dynamics of that PS.
The main constrained resource is the communication bandwidth between
the nodes.

As an example consider an ad-hoc sensor network consisting of
smart-phones. The smart-phones communicate with a network of computers
through cell phone connections and the internet.  The amount of data
generated by a sensor such as a video camera easily exhausts the
bandwidth cellular communication network. Even when the bandwidth is
not exhousted it is usually the most expensive part of operating the
system, both in terms of data-communication costs and in terms of
battery life.

It is therefor desirable to design a system which operates in
such a way as to minimize the amount of communication between the
sensors and the compute nodes. A common approach is to use {\em
  lossless compression}. This is a good solution when possible, but it
rarely decreases the communication volume by a factor bigger than 4.
Our goal is to design method that will decrease the communication
volumes by a factor of ten or more.

To achieve such rates we need to look towards {\em lossy compression}
methods. When data is compressed and decompressed using a lossy
compression method, the result is a {\em distorted} version of the
original data. We say that a compression method is good if a small
data {\em rate} (i.e. the bandwidth required to carry the compressed
data) is enough to achieve low expected {\em distortion}. The
foundational theory of lossy compression is Shannon's Rate-Distortion
theory, which characterizes the achievable rate-distortion pairs.

The reason that lossy compression methods are effective is that not
all of the bits in the digital representation of the sensor
measurements are equally important. A common model of real-world
mesurements considers them to be the sum of a {\em signal} and {\em
  noise}. Where the signal represents the physical quantity measured
by the sensor while noise corresponds to the undesirable external 
and internal influences that cause errors in the
measurements. Noise is often modeled as ``Gaussian White Noise''
(which can be formalized by Using Weiner processes). A raw signal $f$
is usually represented as
\[
f(t) = s(t)+\sigma(t) w(t)
\]
where $s(t)$ is the signal as a function of time $w(t)$ is white
noise and $\sigma(t)$ is the amplitude of the noise.

It is interesting to note that the Shannon information content of
white noise is typically much larger than that of the signal. As a
result, most of the bits used in a lossless compression of the raw
signal will be devoted to the noise! Lossy compression achieves two
things at once: a better compression ratio than lossless compression
and a reconstructed signal that is a cleaner version of the raw
signal.

The idea of partitioning data into signal and noise, which is common
practice in signal processing. Can be generalized by using the concept
of The Kolmogorov Sufficient Statistic. We describe this notion in
detail below. For this introduction, we present a high level
view. Using the method of Kolmogorov sufficient statistic one can
represent any signal using a two part code $(A,r)$ where $A$ is a
computer program and $r$ is an {\em uncompressible} binary
sequence. In the Kolmogorov view of randomness, an uncompressible
sequence is equivalent to a random sequence. We can therefor equate
the program with the signal and $r$ with the noise. While changing $r$
will definitely change the reconstructed sequence, it will change it
in a way that is {\em unconsequential} from a statistical point of
view. We can therefor encode the sequence $s$ using $A$ and will not
be losing any statistically significant information.
 
\iffalse
For now we give an intuitive description. The better
known concept of Kolomogorov complexity quantifies the complexity of a
binary sequence by the length of the shortest program that generates
this sequence.  To define the Kolmogorov Sufficient Statistic we
consider encodings that consist of two parts: a program and an
index. The program is one which can generate any sequence $s$ in a set $S$
of sequence. The index points to a particular sequence in the set. As
the size of the set $S$ is $|S|$, the index requires $\log_2(|S|)$
bits. The algorithm recieves as input the index and it outputs $s$.
We define the 
\fi

Our plan is therefor to partition the signal from a sensor (or a set
of sensors) into the useless and uncompressible noise part and the
useful and compressible signal and noise-amplitude part.

\section{Approximate Views as an abstract data type}

Here is an attempt to formalize the relationship between a physical
system which is continuous in time and space, the discrete sensor data
extracted from it, fitting a model to this data, compressing and
decompressing.
\begin{enumerate}
\item A {\em coordinate system} is the $d$ dimensional Euclidean space
  $R^d$ with agreed upon names for the coordinates. Time is always a
  coordinate. Other possible coordinates define the {\em state} which
  includes quantities such as location, temperature, angle, speed etc. 
\item The {\em raw data} is a database table which we denote by
  $X$. Each row in the table represents a single measurement. The raw
  has time as one of the fields. We denote the {\em Schema} of $X$ by $S$.
\item An {\em approximate view} or a {\em model} $M$ of $X$ is a
  representation of a distribution over instances of the Schema $S$.
\item The model is represented by a program which, given a binary
  sequence, generates a table according to the schema $S$.
\item We say that the model is accurate if there is no program that
  can distinguish between the original table $X$ and a table $X$
  generated by feeding the model with a random binary
  sequence.\footnote{This definition depends on the computational
    model, the ideal is to use the Kolmogorov Definition, but that is
    not practical. A more practical approach would be to use
    computational models of pseudo-randomness (Russel). The truly
    practical methods would be domain-specific and will probably use
    statistical tests or notions of distortion (a-la lossy
    compression)}
\item The ability to generate a table that is statistically
  indistinguishable from the original table is the basic requirements
  that all views have to satisfy. The basic protocol is: the view
  generates a randomized table, and the user computes whatever they
  need to from that table. However, in many cases it is faster to
  compute the answers to queries directly from the model $M$, without
  generating a random sample. The view can provide a mechanism to
  answer such queries directly.
\end{enumerate}

Example: Suppose the raw data is the GPS data from a cellphone
together with some measurements such as pollution level. A model of
the data can be an ARMA model with a stream of correction
vectors. This arma model represents a smoothed and compressed version
of the raw data. A query can be of the form: ``Where was person A at
time $t$?''  Note that the particular time might not exist in the raw
data. The ARMA model generates an answer using interpolation.

The approximate view serves three purposes: Compression, noise
reduction and interpolation (providing values at times where no values
were measured).

\section{Kolmogorov Sufficient Statistic}

We suppose that all of the sensors share a synchronized clock $t$.
We think of the data collected from sensor $i$ as a data stream:
\[
s^i(\tau_1,\tau_2)=[(t^i_1,x^i_1),\ldots,(t^i_n,x^i_n)]
\mbox{ where } \tau_1\leq t^i_1 < t^i_2<\cdots < t^i_n \leq \tau_2
\]
We suppose that $i \in \{1,\ldots,N\}$ and that the $N$ sensors are
sampled during the same time period, although not at the same time
points. We assume that the times and the measurements are real numbers
represented by fixed precision binary representations.

In other words We can represent the collection of measurements
$(s^1(\tau_1,\tau_2),\ldots,s^N(\tau_1,\tau_2)$ as a finite length
binary sequence. We refer to this binary sequence as the {\em raw
  data}.

We assume that the raw data consists of a {\em signal} which
corresponds to properties of the physical system being observed and
{\em noise} which contains no useful information. In practice, the
partition between signal and noise might depend on the the goal of the
system. However, if computation is unbounded, we can use the concept
of the {\em Kolmogorov Sufficient Statistic} to get a
context-independent partition of any binary sequence into signal and
noise.

We briefly describe the Kolmogorov Sufficient Statistic, for a more
in-depth description see~\cite{CoverThomas}, page 175. We fix a
universal turing machine $U$. The encoding of the binary sequence
$x^n$ is a concatanation of two parts: a program $P$ and a binary
string $R$. The requirement is that the program $P$, given the string
$R$ as input, outputs the sequence $x^n$, the length of the encoding
is the sum of the lengths of the two parts: $l(P)+l(R)$.  The
Kolmogorov Structure Function $K_k(x^n|n)$ is defined to be the length
of the shortest input $R$ such that there exists a program of length
at most $k$ which generates $x^n$ upon recieving the input $R$.

It is pretty obvious that one can always transfer a prefix of length
$j$ the input $R$ into the program $P$, there by making the program
longer by $j$ bits and the input shorter by $j$ bits. Therefor
increasing $k$ by one bit decreases  $K_k(x^n|n)$ by at least one bit,
which keeps the total length of the encoding unchanged. Continuing
this way until the length of $R$ is zero we arrive at the standard
definition of the kolmogorov complexity of a sequence.

We are therefor interested in the shortest program which achieves the
kolmogorov complexity (technically: up to a small additive constant).
This program is called the {\em Kolmogorov Sufficient Statistic}. One
can say that the program $P$ captures all of the structure of the data
sequence $x^n$ while the sequence $R$ captures the unstructured or
{\em random} part.

As we consider data streams, it is natural to extend the two part
description given above into a three part description. Our suggestion
is that the encoding consists of a {\em program} $P$, {\em parameters}
$G$, and random $R$. The program $P$ recieves as input both the
parameters $G$ and the input $R$ and it generates data $x^n$. The
difference from the prvious definition is that the length of $P$ does
not depend on $n$, in other words, the program captures the part of
the structure that is independent of the length of the data $x^n$. As
$n$ increases, $G$ is defined in the same way as the Kolmogorov
sufficient statistic but with the constraint that the first part of
the program is $P$. $R$ is the same as before.

This three-part coding allows us to capture stochastic sequences with
non-stationary distributions, hierarchical clustering models, variable
length markov models etc.

\end{document}
